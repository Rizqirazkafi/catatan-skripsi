\documentclass[10pt,]{report}
\usepackage[utf8]{inputenc}
\usepackage{pxfonts}
\usepackage{graphicx}
\usepackage[none]{hyphenat}
\usepackage{ragged2e}
\usepackage{titlesec}
\usepackage{indentfirst}
\usepackage{changepage}
\tolerance=10000
\sloppy
\graphicspath{{images/}}
\usepackage[a5paper, left=2.3cm, right=2cm, top=2.3cm, bottom=2cm, bindingoffset=6mm]{geometry}
\titleformat{\chapter}[display]{\centering\normalfont\fontsize{12}{12}\selectfont\bfseries}{\thechapter}{12pt}{\fontsize{12}{12}\selectfont}
\titlespacing*{\chapter}{0pt}{0pt}{10pt}
\renewcommand\thesection{\Alph{section}.}

\titleformat{\section}[block]{\normalfont\bfseries\fontsize{12}{12}}{\thesection}{10pt}{\fontsize{10}{10}\selectfont}
\titleformat{\chapter}[block]
  {\normalfont\bfseries\filcenter}
  {BAB \Roman{chapter} \break}
  {0em}
  {}
\newcommand{\chapterNoBab}[1]{
  \par\refstepcounter{chapter}% Menambahkan counter section tapi tidak menampilkannya
  \section*{#1}% Membuat section tanpa nomor
  \addcontentsline{toc}{chapter}{\protect\numberline{\thesection}#1}% Opsi untuk menambahkan ke ToC
}
\titlespacing*{\section}{0pt}{0pt}{10pt}


\begin{document}
\begin{titlepage}
	\begin{center}
    {\large \textbf{ANALISIS PERBANDINGAN OTOMASI SISTEM OPERASI GNU/LINUX ANTARA ANSIBLE DENGAN NIXOS MENGGUNAKAN PENDEKATAN DEKLARATIF\\}}
    \vspace{0.5cm}

    \textbf{PROPOSAL SKRIPSI}
    \vfill
    \includegraphics[width=0.5\textwidth]{images/unesa.jpg}\\
    \vspace*{1cm}
    Disusun oleh:\\
    \textbf{M. Rizqi R (20051204034)}\\
    Dosen Pembimbing:\\
    \textbf{Agus Prihanto, S.T., M.Kom.}\\
    \vfill
    {\large \textbf{UNIVERSITAS NEGERI SURABAYA\\ FAKULTAS TEKNIK \\ PROGRAM STUDI TEKNIK INFORMATIKA \\ 2024}}
	\end{center}
\end{titlepage}

\chapter*{KATA PENGANTAR}
\begin{justify}
	Puji syukur kepada Tuhan Yang Maha Esa yang telah melimpahkan rahmat dan
	hidayah-Nya sehingga proposal penelitian dengan judul “Analisis Perbandingan
	Otomasi Sistem Operasi GNU/Linux Antara Ansible dengan NixOS” ini
	dapat terselesaikan. Penulis juga mengucapkan terima kasih kepada seluruh
	yang telah membantu dalam pembuatan proposal penelitian ini.
	\begin{enumerate}
		\item Kedua orang tua atas segala bantuan, bimbingan, dorongan serta doa restu yang diberikan.
		\item Bapak Agus Prihanto, S.T., M. Kom. selaku dosen pembimbing yang telah memberikan arahan dan bimbingan dalam penyusunan proposal ini,.
		\item Teman-teman Mahasiswa yang telah membantu dalam pengumpulan data dan informasi
		\item Universitas Negeri Surabaya yang telah menyediakan fasilitas dan sarana prasarana yang diperlukan dalam penyusunan proposal ini.
	\end{enumerate}
	Kami menyadari bahwa proposal penelitian ini masih jauh dari sempurna. Oleh karena itu,m kami mengharapkan kritik dan saran yang membangin dari para pembaca.
	Akhir kata, kami berharap proposal penelitian ini dapat bermanfaat bagi para pembaca.
\end{justify}
\begin{FlushRight}
	penulis
\end{FlushRight}
\chapter{PENDAHULUAN}
\section{Latar Belakang}
\begin{adjustwidth}{0.70cm}{}
  \vspace{-3mm}
  \hspace{\parindent}
  Pentingnya melakukan manajemen konfigurasi dengan tujuan untuk menghindari
  penulisan manual konfigurasi sistem operasi secara manual setiap kali
  menyiapkan sistem operasi. Manajemen konfigurasi juga digunakan untuk
  mencapai konsistensi dalam setiap kali penerapan sehingga hasil akhir yang
  diinginkan akan sama setiap kali dilakukan. 

  Terdapat beberapa alat untuk melakukan manajemen konfigurasi sistem operasi,
  terutama untuk sistem operasi berbasis GNU/Linux. Ansible dan NixOS menjadi
  dua dari banyak pilihan untuk melakukan manajemen sistem operasi. Keduanya
  memiliki tujuan memudahkan proses manajemen konfigurasi dan memastikan hasil
  akhir yang diinginkan akan sama setiap kali eksekusi.

  Ansible adalah perangkat lunak otomatisasi TI baris perintah yang ditulis
  dalam bahasa Python. Aplikasi ini dapat mengonfigurasi sistem, menerapkan
  perangkat lunak, dan mengatur alur kerja tingkat lanjut untuk mendukung
  penerapan aplikasi, pembaruan sistem, dan banyak lagi (RedHat, 2022a). 
  
  Ansible memungkinkan kita mendeklarasikan konfigurasi sistem kita dalam
  sebuah Ansible Playbook. Ansible Playbook akan dijalankan pada sebuah sistem
  yang telah memiliki sistem operasi. Konfigurasi Ansible Playbook ditulis
  menggunakan format yml yang merupakan format khusus untuk konfigurasi baik
  sistem }aupun aplikasi. Ansible akan menjalankan setiap perintah pada sistem
  operasi yang telah di install secara otomatis satu per satu. Ansible
  memungkinkan kita melakukan setup banyak sistem sekaligus dengan konfigurasi
  yang telah ada. Diharapkan dari Ansible adalah sistem-sistem yang terdaftar
  memiliki hasil akhir yang sama. 

  NixOS adalah sistem operasi berbasis GNU/Linux yang dibangun dengan Nix build
  system. NixOS menggunakan file dalam format “.nix” yang disebut sebagai NixOS
  module untuk mendeklarasikan sebuah sistem. Dalam file tersebut terdapat
  seluruh konfigurasi sistem mulai dari bootloader, packages, users, system
  services.
  
  Apa yang tertulis dalam module tersebut adalah manifestasi dari sistem yang
  dideklarasikan menggunakan bahasa nix yang merupakan bahasa pemrograman
  fungsional. Ini menghasilkan konfigurasi sistem operasi yang reproducible
  sehingga dapat digunakan berkali-kali pada waktu yang berbeda dan
  menghasilkan manifestasi yang tetap. 

  Banyak kelebihan dan beberapa kekurangan yang dimiliki oleh metode deklaratif
  dari NixOS dan metode campuran (imperatif dan deklaratif) dari Ansible.
  Berdasarkan landasan tersebut, maka penulis ingin meneliti dan membandingkan
  baik dari segi performa yang dimiliki oleh masing-masing alat manajemen
  konfigurasi.
\end{adjustwidth}
\vspace{3mm}
\section{Rumusan Masalah}
\vspace{-3mm}
\begin{adjustwidth}{0.70cm}{}
Adapun rumusan masalah berdasarkan latar belakang diatas yaitu:
\begin{enumerate}
  \item Bagaimana membuat sistem GNU/Linux yang terdeklarasi menggunakan alat
    manajemen konfigurasi.
  \item Bagaimana performa alat manajemen konfigurasi yang diimplementasikan.
\end{enumerate}
\end{adjustwidth}
% \input{chapters/pendahuluan.tex}
\section{Tujuan}
\vspace{-3mm}
\begin{adjustwidth}{0.70cm}{}
  \begin{enumerate}
    \item Membuat sistem GNU/Linux yang terdeklarasi menggunakan NixOS dan 
      Ansible
    \item Mengukur performa \it{time execution} alat manajemen konfigurasi
      NixOS dan Ansible
  \end{enumerate}
\end{adjustwidth}
\vspace{3mm}
\section{Manfaat}
\vspace{-3mm}
\begin{adjustwidth}{0.70cm}{}
  Manfaat yang dapat dihasilkan dari penelitian ini antara lain:
  \begin{enumerate}
    \item Efisiensi waktu dalam melakukan manajemen konfigurasi sistem operasi berbasis GNU/Linux.
    \item Konfigurasi sebuah sistem menjadi deklaratif dalam baris kode.
    \item Mengurangi terjadinya human error dalam mengerjakan konfigurasi dan tugas yang berulang-ulang.
    \item Mendapatkan hasil akhir yang konsisten dari penerapan alat manajemen konfigurasi.

  \end{enumerate}
  
\end{adjustwidth} 
\chapter*{BAB II KAJIAN PUSTAKA}
\end{document}
